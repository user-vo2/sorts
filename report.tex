\documentclass[a4paper,12pt,titlepage,finall]{article}

\usepackage[T1,T2A]{fontenc}     % форматы шрифтов
\usepackage[utf8x]{inputenc}     % кодировка символов, используемая в данном файле
\usepackage[russian]{babel}      % пакет русификации
\usepackage{tikz}                % для создания иллюстраций
\usepackage{pgfplots}            % для вывода графиков функций
\usepackage{geometry}		 % для настройки размера полей
\usepackage{indentfirst}         % для отступа в первом абзаце секции
\usepackage{multirow}            % для таблицы с результатами

% выбираем размер листа А4, все поля ставим по 3см
\geometry{a4paper,left=30mm,top=30mm,bottom=30mm,right=30mm}

\setcounter{secnumdepth}{0}      % отключаем нумерацию секций

\usepgfplotslibrary{fillbetween} % для изображения областей на графиках

\begin{document}
% Титульный лист
\begin{titlepage}
    \begin{center}
	{\small \sc Московский государственный университет \\имени М.~В.~Ломоносова\\
	Факультет вычислительной математики и кибернетики\\}
	\vfill
	{\Large \sc Отчет по заданию №1}\\
	~\\
	{\large \bf <<Методы сортировки>>}\\
	~\\
	{\large \bf Вариант 3 / 1 / 2 / 3}
    \end{center}
    \begin{flushright}
	\vfill {Выполнил:\\
	студент 102 группы\\
	Воробьев~Е.~Р.\\
	~\\
	Преподаватель:\\
	Сенюкова~О.~В.}
    \end{flushright}
    \begin{center}
	\vfill
	{\small Москва\\2020}
    \end{center}
\end{titlepage}

% Автоматически генерируем оглавление на отдельной странице
\tableofcontents
\newpage

\section{Постановка задачи}

В данном задании требуется:
\begin{itemize}
\item реализовать на языке Си два различных метода сортировки по неубыванию массива, состоящего из чисел двойной точности (типа double): методом простого выбора и методом Шелла в виде двух отдельных функций типа void,
\item сравнить количество сравнений и перестановок элементов массива для каждой из этих сортировок (в отсортированном по неубыванию массиве, в отсортированном по невозрастанию массиве, в массиве из чисел, полученных случайным образом),
\item проанализировать каждый из способов на эффективность,
\item сделать выводы об их эффективности.
\end{itemize}


\newpage

\section{Результаты экспериментов}
\subsection{Сортировка методом простого выбора} 

Этот прием основан на следующих принципах:
\begin{enumerate} 
  \item Выбирается элемент с наименьшим ключем.
  \item Он меняется местами с первым элементом.
  \item Затем этот процесс повторяется с оставшимися ${n-1}$ элементами, ${n-2}$ элементами и т. д. до тех пор, пока не останется один самый большой элемент.
\end{enumerate}\par
  Число сравнений ключей (С), очевидно, не зависит от начального порядка ключей.
$$C = {\frac{(n-1)n}{2}}$$\par
Обмен элементов происходит только во внешнем цикле, следовательно, число обменов не превосходит ${n-1}$ (худший случай). В лучшем случае (упорядоченный массив) количество обменов равно 0.\par
  Алгоритм просматривает массив, сравнивая каждый элемент с только что обнаруженной минимальной величиной; если он меньше первого, то выполняется некоторое присваивание. Вероятность, что второй элемент окажется меньше первого равна ${\frac{1}{2}}$, с этой же вероятностью происходят присваивания минимуму. Вероятность, что третий элемент окажется меньше первых двух, равна ${\frac{1}{3}}$, а вероятность для четвертого оказаться наименьшим - ${\frac{1}{4}}$ и т. д. Поэтому полное среднее число пересылок равно $H_{n - 1}$, где $H_n$ – n-ое гармоническое число
\begin{equation}
H_{n} = 1 + {\frac{1}{2}} + {\frac{1}{3}} + ... + {\frac{1}{n}}
\end{equation}
$H_n$ можно представить как
\begin{equation}
H_{n} = ln(n) + g + {\frac{1}{2n}} - {\frac{1}{12n^2}} + ...
\end{equation}
где $g = 0.577216...$ – константа Эйлера. Для достаточно больших n можно отбросить дробные члены и получить приближенное среднее число присваиваний на ${i}$-м проходе в виде
\begin{equation}
F_{i} = ln(i) + g + 1
\end{equation}
Тогда среднее число пересылок $M_{avg}$ в сортировке выбором равно сумме величин
$F_i$ с i , пробегающим от 1 до n :
\begin{equation}
M_{avg} = n(g+1) + \sum \limits_{i=1}^{n} ln(i)
\end{equation}
Аппроксимируя дискретную сумму интегралом
\begin{equation}
\int_{1}^{n}ln(x) dx = nln(n) - n + 1
\end{equation}
получаем приближенное выражение
\begin{equation}
M_{avg} = n(ln(n) + g)
\end{equation}
\begin{table}[h]
\centering
\begin{tabular}{|c|c|c|c|c|c|c|c|}
    \hline
    \multirow{2}{*}{\textbf{n}} & \multirow{2}{*}{\textbf{Параметр}} & \multicolumn{4}{|c|}{\textbf{Номер сгенерированного массива}} & \textbf{Среднее} \\
    \cline{3-6}
    & & \parbox{1.5cm}{\centering 1} & \parbox{1.5cm}{\centering 2} & \parbox{1.5cm}{\centering 3} & \parbox{1.5cm}{\centering 4} & \textbf{значение} \\
    \hline
    \multirow{2}{*}{10} & Сравнения & 45 & 45 & 45 & 45 & 45\\
    \cline{2-7}
                        & Перемещения & 0 & 5 & 5 & 6 & 4\\
    \hline
    \multirow{2}{*}{100} & Сравнения & 4950 & 4950 & 4950 & 4950 & 4950 \\
    \cline{2-7}
                         & Перемещения & 0 & 50 & 94 & 90 & 58.5 \\
    \hline
    \multirow{2}{*}{1000} & Сравнения & 49950 & 49950 & 49950 & 49950 & 49950\\
    \cline{2-7}
                          & Перемещения & 0 & 500 & 992 & 989 & 620.25\\
    \hline
    \multirow{2}{*}{10000} & Сравнения & 49995000 & 49995000 & 49995000 & 49995000 & 49995000\\
    \cline{2-7}
                           & Перемещения & 0 & 5000 & 9989 & 9987 & 6244\\
    \hline
\end{tabular}
\caption{Результаты работы сортировки методом простого выбора}
\end{table}

\newpage

\subsection{Сортировка методом Шелла} 

Метод, предложенный Дональдом Л. Шеллом, является неустойчивой сортировкой по месту. \par
В лучшем случае (на отсортированном массиве)условие внутри цикла не выполняется ни разу, значит в цикле будет всегда постоянное число итераций. В таком случае сложность фиксирована - $O(nlog(n))$. \par
В наихудшем случае можно оценить число сравнений/перемещений, как $n^2$+${\frac{n^2}{2}}$+${\frac{n^2}{4}}$+${\frac{n^2}{8}}$+...<=$2n^2\in O(n^2)$.\par
Среднее время работы алгоритма зависит от длин промежутков — $d$, на которых будут находиться сортируемые элементы исходного массива ёмкостью $N$ на каждом шаге алгоритма.\par
Первоначально используемая Шеллом последовательность длин промежутков: ${d_{1}={\frac{N}{2}},d_{i}={\frac{d_{i-1}}{2}},d_{k}=1}$ в худшем случае, сложность алгоритма составит $O(N^2)$;

\begin{table}[h]
\centering
\begin{tabular}{|c|c|c|c|c|c|c|c|}
    \hline
    \multirow{2}{*}{\textbf{n}} & \multirow{2}{*}{\textbf{Параметр}} & \multicolumn{4}{|c|}{\textbf{Номер сгенерированного массива}} & \textbf{Среднее} \\
    \cline{3-6}
    & & \parbox{1.5cm}{\centering 1} & \parbox{1.5cm}{\centering 2} & \parbox{1.5cm}{\centering 3} & \parbox{1.5cm}{\centering 4} & \textbf{значение} \\
    \hline
    \multirow{2}{*}{10} & Сравнения & 22 & 35 & 27 & 31 & 28.75
\\
    \cline{2-7}
                        & Перемещения & 0 & 13 & 5 & 9 & 7.25
\\
    \hline
    \multirow{2}{*}{100} & Сравнения & 503 & 763 & 1005 & 920 & 797.75
\\
    \cline{2-7}
                         & Перемещения & 0 & 260 & 502 & 418 & 295
\\
    \hline
    \multirow{2}{*}{1000} & Сравнения & 8006 & 12706 & 15428  & 16331 & 13117.75
\\
    \cline{2-7}
                          & Перемещения & 0 & 4700 & 7422 & 8325 & 5111.75
\\
    \hline
    \multirow{2}{*}{10000} & Сравнения & 120005 & 182565 & 266921 & 278931 & 212105.5
\\
    \cline{2-7}
                           & Перемещения & 0 & 62560 & 146916 & 158926 & 92100.5
\\
    \hline
\end{tabular}
\caption{Результаты работы сортировки методом Шелла}
\end{table}

\newpage

\section{Структура программы и спецификация функций}
\begin{itemize}
\item void swap(double *a, double *b) \par
	Функция меняет местами значения двух переменных, ничего не возвращает.
\item void generate\underline{ }random\underline{ }array(int size, double *array) \par
	 Данная функция заполняет получаемый массив array заданной величины size случайными числами типа double, ничего не возвращает.
\item void generate\underline{ }ordered\underline{ }array(int size, double *array) \par
	Данная функция заполняет получаемый массив array заданной величины size числами от $-\frac{size}{2}$ до $\frac{size}{2} - 1$, ничего не возвращает.
\item void generate\underline{ }reversed\underline{ }array(int size, double *array) \par
	Данная функция заполняет получаемый массив array заданной величины size числами от $\frac{size}{2}$ до $-(\frac{size}{2} + 1)$, ничего не возвращает.
\item void print\underline{ }array(int size, double *array) \par
	Данная функция выводит получаемый массив array размером size, ничего не возвращает.
\item void selection\underline{ }sort(int size, double *array) \par
	Данная функция сортирует получаемый на вход массив array величины size с помощью сортировки методом простого выбора, ничего не возвращает.
\item void shell\underline{ }sort(int size, double *array) \par
	Данная функция принимает на вход массив array размером size. Функция реализует сортировку методом Шелла, в качестве 1-го шага взят предложенный самим Шеллом вариант $shift={\frac{N}{2}}$, в качестве последующих шагов - предыдущий, поделенный целочисленно на 2, посредством битового сдвига. Функция ничего не возвращает.	
\end{itemize}
\newpage

\section{Отладка программы, тестирование функций}
\subsection{Тестирование сортировки методом простого выбора}
\begin{table}[h]
\centering
\begin{tabular}{|c|c|c|c|c|c|c|c|c|c|c|c|}
    \hline
    i & 0 & 1 & 2 & 3 & 4 & 5 & 6 & 7 & 8 & 9\\
    \hline
     array[i](ввод) & -5 & -4 & -3 & -2 & -1 & 0 & 1 & 2 & 3 & 4 \\
    \hline
     array[i](вывод) & -5 & -4 & -3 & -2 & -1 & 0 & 1 & 2 & 3 & 4 \\
    \hline
\end{tabular}
\caption{Тест 1.1 (начальная последовательность отсортирована по неубыванию)}
\end{table}

\begin{table}[h]
\centering
\begin{tabular}{|c|c|c|c|c|c|c|c|c|c|c|c|}
    \hline
    i & 0 & 1 & 2 & 3 & 4 & 5 & 6 & 7 & 8 & 9\\
    \hline
     array[i](ввод) & 4 & 3 & 2 & 1 & 0 & -1 & -2 & -3 & -4 & -5 \\
    \hline
     array[i](вывод) & -5 & -4 & -3 & -2 & -1 & 0 & 1 & 2 & 3 & 4 \\
    \hline
\end{tabular}
\caption{Тест 1.2 (начальная последовательность отсортирована по невозрастанию)}
\end{table}

\begin{table}[h]
\centering
\begin{tabular}{|c|c|c|c|c|c|c|c|c|c|c|c|}
    \hline
    i & 0 & 1 & 2 & 3 & 4 & 5 & 6 & 7 & 8 & 9\\
    \hline
     array[i](ввод) & -16 & -26 & 12 & -5 & -11 & -8 & 41 & 3 & -29 & 45 \\
    \hline
     array[i](вывод) & -29 & -26 & -16 & -8 & -5 & 3 & 11 & 12 & 41 & 45 \\
    \hline
\end{tabular}
\caption{Тест 1.3 (начальная последовательность состоит из случайных чисел)}
\end{table}

\clearpage


\subsection{Тестирование сортировки методом Шелла}

\begin{table}[h]
\centering
\begin{tabular}{|c|c|c|c|c|c|c|c|c|c|c|c|}
    \hline
    i & 0 & 1 & 2 & 3 & 4 & 5 & 6 & 7 & 8 & 9\\
    \hline
     array[i](ввод) & -5 & -4 & -3 & -2 & -1 & 0 & 1 & 2 & 3 & 4 \\
    \hline
     array[i](вывод) & -5 & -4 & -3 & -2 & -1 & 0 & 1 & 2 & 3 & 4 \\
    \hline
\end{tabular}
\caption{Тест 2.1 (начальная последовательность отсортирована по неубыванию)}
\end{table}

\begin{table}[h]
\centering
\begin{tabular}{|c|c|c|c|c|c|c|c|c|c|c|c|}
    \hline
    i & 0 & 1 & 2 & 3 & 4 & 5 & 6 & 7 & 8 & 9\\
    \hline
     array[i](ввод) & 4 & 3 & 2 & 1 & 0 & -1 & -2 & -3 & -4 & -5 \\
    \hline
     array[i](вывод) & -5 & -4 & -3 & -2 & -1 & 0 & 1 & 2 & 3 & 4 \\
    \hline
\end{tabular}
\caption{Тест 2.2 (начальная последовательность отсортирована по невозврастанию)}
\end{table}

\begin{table}[h]
\centering
\begin{tabular}{|c|c|c|c|c|c|c|c|c|c|c|c|}
    \hline
    i & 0 & 1 & 2 & 3 & 4 & 5 & 6 & 7 & 8 & 9\\
    \hline
     array[i](ввод) & 40 & -44 & 14 & -45 & 20 & -49 & -21 & 4 & 16 & -42 \\
    \hline
     array[i](вывод) & -49 & -45 & -44 & -42 & -21 & 4 & 14 & 16 & 20 & 40 \\
    \hline
\end{tabular}
\caption{Тест 2.3 (начальная последовательность состоит из случайных чисел)}
\end{table}

\clearpage

\newpage

\section{Анализ допущенных ошибок}

Ошибок допущено не было.

\newpage
\begin{raggedright}
\addcontentsline{toc}{section}{Список цитируемой литературы}
\begin{thebibliography}{99}
\bibitem{cs} Кормен Т., Лейзерсон Ч., Ривест Р, Штайн К. Алгоритмы: построение и анализ.
    Второе издание.~--- М.:<<Вильямс>>, 2005.
\bibitem{cs} Вирт Н. Алгоритмы и структуры данных. ~--- М.: Мир, 1989
\bibitem{cs} Д. Кнут. Искусство программирования. Том 3. Сортировка и поиск, 2-е изд. ~--- М.:<<Вильямс>>, 2017.
\end{thebibliography}
\end{raggedright}
\end{document}
